\documentclass[a4paper,12pt]{article}
\usepackage[utf8]{inputenc}

%opening
\title{Double pendulum simulator\\User manual}
\author{Bence Göblyös}

\begin{document}

\pagenumbering{gobble}
\maketitle
\newpage
\pagenumbering{arabic}

\section{Installation}

\subsection{Linux and BSD}

\subsubsection{Dependencies}
A C compiler is required to build the application.
It has been tested with GCC (and briefly with clang),
but any other compiler should work as well.
It is also advised to have access to the \texttt{make} utility to run the build scripts.

The application relies on \texttt{gnuplot} to draw the phase space plots,
so it is highly recommended to install it.

ImageMagick may be used to convert the PPM output into more common formats,
but it is only required if the image viewer of choice doesn't support PPM images
(KDE's \texttt{gwenview} can open them by default, for instance).\\\\
Platform-specific install commands:
\begin{itemize}
 \item Debian:\\ \texttt{apt install build-essential gnuplot-nox imagemagick}
 \item Fedora:\\ \texttt{dnf install @c-development gnuplot-minimal ImageMagick}
 \item FreeBSD:\\ \texttt{pkg install gnuplot-lite ImageMagick7-nox11}
\end{itemize}

\subsubsection{Building}

To build the application, run \texttt{make} as an unprivileged user.
In case \texttt{make} is not available,
the following command will compile the application:\\
\texttt{cc -s -O2 src/main.c src/input.c src/sim.c -o bin/dpsim -lm}

\subsubsection{Installing}

The build process will produce a binary in the bin directory.
It is possible to install it to \texttt{/usr/local/bin/dpsim}
by running \texttt{make install} as root.

\end{document}
